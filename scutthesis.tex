% 注:1. pdfcover的内容根据学校要求自定义(参考cover_file文件夹)。2. 这里代码写得不是很完备,第二个参数只影响封面,其他内容如页眉不受影响。还需在下面代码进一步调整。这里还需完善。为简单期间,下面三个选项供选择:
\documentclass[unicode,master]{scutthesis} % 硕士草稿封面,使用正式封面时注释该行
% \documentclass[unicode]{scutthesis} %  博士草稿封面,使用正式封面时注释该行
% \documentclass[unicode,pdfcover]{scutthesis} %   论文正式封面(不使用选项来区分硕博士,而是大家制作自己的封面文件thesis_cover.pdf,盲审时删除必要的封面内容即可),使用草稿封面时注释该行

\usepackage{array,longtable,graphicx} 
\usepackage{anyfontsize} %消除字体警告
\usepackage{enumitem}
%%%%%%%%%%%%%%%%%%%%%%%%%%%%%%%%%%%%%%%%%%%%%%%%%%%%%%%%%%%%%%%%%%%%%%%%%%%%%%%%%%
%编译范围
% \includeonly{chapter/chapter04}
%参考文献设置
\usepackage[backend=biber,style=gb7714-2015,gbalign=gb7714-2015,gbpub=false,gbnamefmt = lowercase]{biblatex}% 可试着设style=gb7714-1987
\addbibresource[location=local]{biblibrary/MyLibrary.bib} % 如果在其他盘,改为相对路径。比如F盘,改为:F/MyLibrary.bib
\addbibresource[location=local]{biblibrary/mybibfile2.bib} % 无论什么来源的bib文件,只要由参考文献的BibTeX组成,都可以使用此模板。参考文献的BibTeX获取方法可百度
%页眉页脚设置
\usepackage{fancyhdr}
\usepackage{listings}
\usepackage{xunicode}
\renewcommand{\lstlistingname}{列表}
\pagestyle{fancy}
\fancyfoot[C]{\headfont\thepage}
\renewcommand{\chaptermark}[1]{\markboth{\chaptername\ #1}{}}
\renewcommand{\sectionmark}[1]{\markright{\thesection\ #1}}
\fancyhead[RE]{}
\fancyhead[RO]{}
\fancyhead[LE]{}
\fancyhead[LO]{}
\fancyhead[CO]{\headfont{\leftmark}}
\fancyhead[CE]{\headfont{华南理工大学硕士学位论文}}% 博士自行修改
\renewcommand{\headrulewidth}{1.5pt}
\renewcommand{\footrulewidth}{0pt}
%%%%%%%%%%%%%%%%%%%%%%%%%%%%%%%%%%%%%%%%%%%%%%%%%%%%%%%%%%%%%%%%%%%%%%%%%%%%%%%%%%
\usepackage[unicode=true,bookmarks=true,bookmarksnumbered=true,bookmarksopen=false,breaklinks=false,pdfborder={0 0 1},backref=false,colorlinks=true]{hyperref}
%  \hypersetup{...}主要影响生成的 PDF 文件的元数据(PDF属性)和超链接样式,不会影响论文正文内容的显示。
\hypersetup{pdftitle={LaTeX模板使用说明},
	pdfauthor={姓名}, % 可能会泄露作者信息,盲审时请注意
	pdfsubject={华南理工大学硕士学位论文}, % 博士自行修改
	pdfkeywords={PDF关键字1;PDF关键字2},
%%		linkcolor=black, anchorcolor=black, citecolor=black, filecolor=black, menucolor=black, urlcolor=black, pdfstartview=FitH}% 黑白,提交版
	linkcolor=blue, anchorcolor=black, citecolor=red, filecolor=magenta, menucolor=red, urlcolor=magenta, pdfstartview=FitH}% 彩色

\makeatletter
%%%%%%%%%%%%%%%%%%%%%%%%%%%%%% LyX specific LaTeX commands.
\providecommand{\LyX}{\texorpdfstring%
	{L\kern-.1667em\lower.25em\hbox{Y}\kern-.125emX\@}
	{LyX}}
%% Because html converters don't know tabularnewline
\providecommand{\tabularnewline}{\\}
\makeatother
\begin{document}
	%%%%%%%%%%%%%草稿封面设置%%%%%%%%%%%%%使用“正式封面”时不需要理会这部分
	\title{面向协同边缘计算的资源联合任务调度系统研究}	
	\author{梁华倩}	
	\supervisor{指导教师:杨磊\ 教授}	
	\institute{华南理工大学}
	\date{2026年3月1日}
	%%%%%%%%%%%%%%%%%%%%%%%%%%%%%%%%%%%%%
	\maketitle	
	\frontmatter	%此后为罗马数字页码,页面类型为plain
	\chapter{摘\texorpdfstring{\quad}{}要}
协同边缘计算作为一种新兴范式,通过利用地理分布且异构的边缘节点资源,为自动驾驶、工业互联网等低延迟、高可靠应用提供了技术支撑。然而,边缘环境的异构性、计算与网络资源的紧密耦合性,以及分布式智能应用(如分布式训练)所呈现的复杂网络依赖拓扑,使得传统的任务调度机制在资源效率和系统性能上面临严峻挑战。当前研究主要存在两大局限:其一,主流基于有向无环图的任务依赖模型难以有效刻画可能具有环状等复杂通信特征的分布式智能任务;其二,现有调度方案普遍将计算资源与网络资源进行独立管理与调度,忽略了二者间的内在权衡关系,容易导致资源竞争与性能瓶颈。为解决上述问题,本文在协同边缘计算环境下设计了一种高效的计算与网络资源联合调度方案,主要工作如下:

本文设计并实现了协同边缘调度系统CES(Collaborative Edge Scheduler),首次将任务到节点的映射与网络带宽分配进行联合管理。其次,构建了一个计算与网络资源的联合优化模型,该模型能够精确描述分布式训练等任务的通信拓扑与资源需求,并弹性适应动态环境。我们进一步提出了一种基于深度强化学习的智能调度算法,采用双分支演员-评论家架构,并融入启发式奖励函数,使系统能够在动态环境中自主学习全局最优的联合调度策略,从而实现系统负载均衡与用户带宽满足度等整体性能的优化。仿真实验和系统实验结果表明,该算法在整体性能上显著优于现有基准方法。

此外,为应对多分布式学习作业并发的场景,本文在CES基础上进一步提出了CES-Multi算法。该算法以提升资源利用均衡度、保障作业带宽需求满足度及降低作业平均等待时间为目标,在边缘节点多资源约束下,融合滚动窗口机制、资源感知贪心排序与动态饥饿保护策略。该算法通过滚动窗口动态选择候选作业,依据资源适配度、业务优先级和等待时间进行综合评分与排序,并调用既有单作业调度器执行资源分配;同时通过优先级动态提升与最小资源预留等机制,避免低优先级作业饥饿。实验表明,CES-Multi在资源利用率、作业等待时间与带宽满足度方面的综合表现优于现有方法,有效增强了CES系统在多作业调度场景中的适用性与综合性能。

\keywordsCN{边缘计算;任务调度;多目标优化;深度强化学习;分布式训练}

\chapter{Abstract}
Collaborative edge computing, as an emerging paradigm, provides technical support for low-latency and high-reliability applications such as autonomous driving and industrial internet by leveraging geographically distributed and heterogeneous edge node resources. However, the heterogeneity of edge environments, the tight coupling between computing and network resources, and the complex network-dependent topologies exhibited by distributed intelligent applications (e.g., distributed training) pose severe challenges to traditional task scheduling mechanisms in terms of resource efficiency and system performance. Current research mainly suffers from two limitations: First, mainstream task dependency models based on directed acyclic graphs (DAGs) struggle to effectively characterize distributed intelligent tasks that may have complex communication patterns such as cycles. Second, existing scheduling schemes generally manage and schedule computing resources and network resources independently, overlooking the inherent trade-off between them, which can easily lead to resource contention and performance bottlenecks. To address the above issues, this paper designs an efficient joint scheduling scheme for computing and network resources in a collaborative edge computing environment. The main contributions are as follows:

This paper designs and implements a collaborative edge scheduling system named CES (Collaborative Edge Scheduler), which, for the first time, jointly manages task-to-node mapping and network bandwidth allocation. Secondly, a joint optimization model for computing and network resources is constructed, which can accurately describe the communication topology and resource requirements of tasks such as distributed training and elastically adapt to dynamic environments. We further propose an intelligent scheduling algorithm based on deep reinforcement learning, which adopts a dual-branch actor-critic architecture and incorporates a heuristic reward function, enabling the system to autonomously learn the globally optimal joint scheduling policy in dynamic environments, thereby optimizing overall performance metrics such as system load balancing and user bandwidth satisfaction. Simulation and system experimental results demonstrate that the proposed algorithm significantly outperforms existing baseline methods in overall performance.

Furthermore, to address scenarios with concurrent multiple distributed learning jobs, this paper proposes the CES-Multi algorithm based on CES. Aiming to improve resource utilization balance, guarantee job bandwidth satisfaction, and reduce average job waiting time, this algorithm integrates a rolling window mechanism, resource-aware greedy sorting, and dynamic starvation prevention strategies under multi-resource constraints of edge nodes. The algorithm dynamically selects candidate jobs through a rolling window, performs comprehensive scoring and sorting based on resource suitability, business priority, and waiting time, and invokes the existing single-job scheduler for resource allocation. Meanwhile, mechanisms such as dynamic priority boosting and minimum resource reservation are employed to avoid starvation of low-priority jobs. Experiments show that CES-Multi outperforms existing methods in terms of resource utilization, job waiting time, and bandwidth satisfaction, effectively enhancing the applicability and comprehensive performance of the CES system in multi-job scheduling scenarios.

\keywordsEN{Edge Computing; Task Scheduling; Multi-objective Optimization; Deep Reinforcement Learning; Distributed Training} % 中英文摘要
	%%%%%%%%%%%%%%%%%%%%%%%%%%%%%%%%%%%%%%%%%%%%%%%%
	% 目录、表格目录、插图目录这几个字本身的大纲级别是一级的,即和章名有相同的字号字体。目录表的内容通过titletoc宏包在。cls文件设置了。
	%\cleardoublepage % pdfbookmark可能需要这一条才能正常工作
	\pdfbookmark{\contentsname}{toc} %为目录添加pdf文件书签
	\tableofcontents	%目录
    %插图目录(可选)、 表格目录(可选)
	\begingroup
		\renewcommand*{\addvspace}[1]{}
		\newcommand{\loflabel}{图} 
		\renewcommand{\numberline}[1]{\loflabel~#1\hspace*{1em}}	
		\listoffigures
		
		% \newcommand{\lotlabel}{表}
		% \renewcommand{\numberline}[1]{\lotlabel~#1\hspace*{1em}}
		% \listoftables
	\endgroup

	%%%%%%%%%%%%%%%%%%%%%%%%%%%%%%%%%%%%%%%%%%%%%%%%%
	% \include{chapter/symbols}	% 符号对照表(可选)
	% \include{chapter/abbreviation} 	% 缩略词	
	
	\mainmatter %此后为阿拉伯数字页码
	
    %%%%%%%%%%%%%%%%%%%%%%%%%%%%%%%%%%%%%%%%%%%%%%页眉页脚设置  
    \fancypagestyle{plain}{
    	\pagestyle{fancy}
    }	% 每章的第一页会默认使用plain,没有页眉。通过重定义plain为fancy解决
    \pagestyle{fancy}	%设置页眉页脚为fancy
    %%%%%%%%%%%%%%%%%%%%%%%%%%%%%%%%%%%%%%%%%%%%%%分章节,结合导言区的\includeonly命令可仅编译部分章节,编译时不用切换界面,直接在相应章节编译即可。
	\chapter{绪论}
本章为本课题提供了简要介绍。首先概述了研究背景和意义,然后总结了国内外关于协同边缘计算下分布式训练任务调度方法的研究现状,进而详细阐述了本文的主要研究工作及创新点。最后,对本文的组织结构进行了说明。
\section{研究背景和意义}
近年来,随着物联网与人工智能技术的深度融合与规模化应用,计算范式正经历一场深刻的变革,从以云计算为中心的集中式处理模式,逐步向“云-边-端”协同的分布式智能架构演进。在这一过程中,边缘计算作为关键一环,通过将计算、存储与分析能力下沉至网络边缘侧,直接在数据源头或近端进行实时处理,有效缓解了云中心在带宽消耗、服务延迟和隐私安全方面的巨大压力。然而,单一、孤立的边缘节点往往受限于其固有的资源瓶颈(如算力、存储)与物理覆盖范围,难以独立支撑日益复杂、跨地域协作且对实时性有严苛要求的智能应用。在此背景下,协同边缘计算应运而生,它超越了对单一边缘节点的优化,演进为一种旨在实现地理分布广泛、形态异构(包括边缘服务器、网关、车载单元及各类移动设备)的众多边缘节点之间,进行数据、计算、模型与网络资源深度共享与协同调度的新兴范式。该架构通过一个统一的管控平面,对大规模、分布广泛且异构的边缘基础设施进行联合管理与智能编排,从而将离散的边缘节点组织成一张高效的边缘协同网络。这不仅能够极大地提升系统在服务可靠性、资源利用率和任务完成效率等方面的表现,支持业务的灵活部署与快速规模化扩展,更推动了无处不在的智能服务真正走向现实。

在上述发展趋势中,分布式训练是实现边缘智能的关键任务,广泛出现在智能驾驶(多车感知协同)、工业互联网(跨厂区质量检测)和智慧城市(分布式视频分析)等前沿领域。此类任务通常将训练数据分散在多个边缘节点上,通过协同执行本地计算并频繁交换模型参数或梯度信息,共同完成全局模型的更新。在协同边缘计算环境下,分布式训练任务的执行涉及多个边缘节点之间的紧密协作,形成一个包含数据并行或模型并行的计算—通信流程。在协同边缘计算环境下,这样一个分布式训练任务的执行,本质上构建了一个逻辑上紧密耦合、物理上分散的计算-通信协同体。其性能表现呈现出鲜明的双重依赖性:既依赖于各参与节点的本地计算能力,更严重依赖于节点间通信链路的带宽、延迟和稳定性。任务的逻辑通信拓扑(如环形、星形等)如何高效、低冲突地映射到底层物理网络拓扑上,直接决定了同步过程的通信开销,进而成为影响整体训练效率(如达到目标精度所需的时间)的决定性瓶颈。


这一将分布式训练等复杂任务调度到协同边缘网络中的场景,对资源管理和任务编排提出了一定的挑战。具体而言,该调度场景的核心特征体现在多维资源的耦合性与任务需求的复杂性上。首先,边缘节点并非孤立的计算单元,其计算资源(如CPU、内存)与网络资源(如带宽)紧密耦合,任务对一种资源的消耗往往会影响另一种资源的可用性。其次,任务内部(如分布式训练中的工作节点之间)存在着严格的数据依赖与同步点,形成了复杂的通信拓扑。任务的性能不仅由单个节点的计算能力决定,还会显著受到其内部通信链路所经历的物理网络性能的影响。任务的通信拓扑与底层物理网络拓扑的匹配程度,直接决定了通信开销的大小,从而成为系统整体性能的关键瓶颈。这些因素共同带来了如下核心难点:

(1)分布式训练等任务产生的密集、周期性通信流,在映射到共享的物理网络时,会竞争有限的带宽资源。由于任务逻辑通信拓扑可能非常复杂,不同任务间甚至同一任务内的数据流可能在不经意间共享某条关键物理链路,形成隐蔽的带宽竞争点。这种竞争难以通过局部信息进行准确建模和预测,极易引发局部网络拥塞,造成部分任务同步延迟激增,从而破坏系统整体的负载均衡与性能稳定性。

(2)调度器在决策时面临固有的目标权衡。例如,为了最小化通信开销,理想策略是将频繁通信的任务子单元(如一个分布式训练的Worker)尽可能集中部署在物理距离近、带宽充足的少数节点上。但这往往会导致这些节点形成计算热点,引发排队延迟,并违背了负载均衡的原则。反之,为了最大化计算资源的均衡利用,将任务分散部署到更广泛的节点上,又会显著增加节点间的通信距离与跳数,从而加大通信延迟和带宽压力。这种“计算聚集”与“通信分散”之间的矛盾,使得在协同边缘计算中实现计算与网络资源的联合均衡调度成为一个难以同时优化所有目标的复杂问题。

综上所述,在协同边缘计算环境下,如何设计一种能够同时考虑计算资源与网络资源,并在二者之间实现智能权衡的调度机制,具有广阔的应用前景及重要的研究意义。

\section{研究现状与分析}
现有的边缘计算任务调度系统仍存在诸多局限:FlexiTask虽然将带宽纳入Kubernetes边缘环境的资源模型,并结合负载预测提升效率,但其视带宽为静态资源,忽略任务间网络依赖,优化目标局限于节点级负载均衡。Edge Service框架支持QoS感知调度,但缺乏网络联合调度与长期全局优化能力,且网络模型过于简化。FAOFE采用有向无环图(DAG)对任务依赖进行建模,但未考虑网络状态,依赖静态规则,难以响应动态变化。ENTS虽深化了网络资源集成并提升吞吐量,但仍未充分实现计算与网络资源的协同优化,且存在目标单一或模型简化问题。

当前边缘计算任务调度算法研究普遍采用有向无环图(DAG)对任务建模,该模型适用于具有任务依赖关系的任务流,并在工作流调度等场景中表现良好。然而,随着边缘智能与分布式协同计算的发展,分布式训练等应用呈现出环状拓扑和动态交互特征,难以用DAG准确描述,导致现有调度机制在适应性和资源效率方面存在明显不足。另一方面,虚拟网络嵌入(VNE)的研究多聚焦于节点计算与内存资源优化,常将网络资源视为次要约束。这种忽略带宽和拓扑调度的做法在多租户、动态变化的边缘环境中易引发资源竞争和传输瓶颈,已成为系统性能的主要短板。
\section{本文研究工作及创新点}
针对上述不足,本文以分布式训练等网络依赖型任务为应用背景,旨在设计并开发一种能够联合调度计算与网络资源的任务调度系统。不再将“任务如何部署”和“网络带宽分多少”当作两个独立的问题,而是作为一个统一的问题来求解。我们构建了同时考虑计算资源和网络资源的联合优化模型,并基于深度强化学习生成全局最优调度策略。该系统致力于为动态、异构的边缘环境提供资源感知能力强、整体性能优越且具备自适应能力的任务—资源协同管理机制,以弥补现有研究在资源协同方面的缺陷。本文的主要贡献如下:

(1)本文首次提出了面向计算资源与网络资源联合调度的 CES 系统:我们设计并实现了一个名为 CES 的系统,能够在分布式边缘基础设施中协同完成计算任务到节点的映射与网络带宽资源的动态分配,实现两类资源的统一管理与调度。

(2)本文构建了计算与网络资源的联合优化模型:建立了同时考虑计算资源(CPU、内存)和网络资源(带宽)的联合优化模型。该模型能够准确描述分布式学习等通信依赖型任务在边缘环境中的通信与计算需求,并弹性适应动态带宽变化,从而实现任务映射策略与带宽分配策略的协同优化。

(3)本文设计了基于强化学习的智能调度机制:采用深度强化学习方法,设计了双分支演员-评论家结构,在奖励函数设计中融入启发式思想,帮助智能体快速学习决策,以实现任务部署与网络资源分配的联合决策。

(4)本文通过仿真实验和系统实验在三种分布式学习任务场景下测试了CES的智能调度机制性能。相较于基准算法,CES能够更好地平衡负载均衡度和带宽满足度,显著提高整体性能。
\section{本文组织结构}


关于\LaTeX{}以及基于\LaTeX{}写作的好处不再赘述。\LaTeX{}的入门资料推荐文献\parencite{_g}以及文献\parencite{_c}。 %第一章
	\chapter{相关技术基础概述}
%
\section{协同边缘计算}
\section{分布式训练}
\section{任务调度}
\section{深度强化学习}
\section{本章小结}

%第二章
	\chapter{协同边缘计算任务调度系统CES设计}
本节将详细阐述CES协同边缘计算调度系统的设计思路、核心架构与关键流程。首先,本节将明确系统旨在解决的核心问题与设计目标;其次,将深入剖析基于主从架构的系统组件设计与交互机制;最后,将系统地描述一个作业从提交到执行完毕的任务调度过程。

\section{系统设计目标}
CES协同边缘计算调度系统是一种面向智能边缘环境的任务调度系统,旨在解决弹性环境中具有通信依赖关系的分布式任务调度问题。此处定义的“弹性环境”包含双重动态性:一是计算与网络物理资源可随节点加入或离开而动态变化;二是待处理的任务规模与资源需求可随时间动态波动。在此复杂环境下,传统的、孤立考虑计算或网络资源的调度策略难以实现整体效能最优。

因此,本系统的核心设计目标是实现对计算资源与网络资源的统一感知与联合调度。系统需构建一个全局的、融合的资源视图,统一纳管任务数据布局、节点计算能力与网络链路状态,并以此为基础进行协同决策。具体而言,CES系统旨在达成以下三个关键设计指标:
\begin{itemize}
\item \textbf{确保系统在真实边缘环境下的调度可信度}:系统需直接部署并运行于真实的边缘计算环境中,该环境天然具备节点的异构性(如计算能力与网络接口的差异)、资源的动态弹性以及网络拓扑的复杂性。调度系统必须能够在此类真实、动态的条件下,持续感知环境变化,并生成与执行有效的调度决策。其目标是保证系统在实际部署中的调度行为与结果真实、可靠,从而验证其工程实用性与有效性。

\item \textbf{提升跨边缘节点的资源利用均衡度}:避免部分节点过载而其他节点空闲的资源碎片化现象。调度算法应综合考虑CPU、内存及网络等多维资源,力求在系统全局范围内实现负载均衡,从而提高基础设施的整体资源利用率与投资回报率。

\item \textbf{保障作业间的性能隔离}:在共享的、多租户边缘环境中,确保不同作业的运行环境相互隔离,避免资源争用导致的性能干扰。系统需通过容器化技术与细粒度的资源配额管理,强制实施CPU、内存及网络带宽的资源隔离,为共存的作业提供独立、可控的执行环境,保障其性能的独立性与可预期性。
\end{itemize}

为实现上述目标,CES系统被设计为一个覆盖“资源感知-智能决策-精准执行”的任务调度系统。该系统不仅强调调度算法在理论上的优化能力,更注重其在实际异构、动态边缘环境中的工程可实现性与鲁棒性,从而最终为用户提供一个稳定、高效且资源感知的边缘计算服务基底。

\begin{figure}[htbp]
	\centering
	\includegraphics[scale=0.45]{Fig/ces-cn_system_framework.png}
    \caption{\label{system_framework}协同边缘任务调度系统CES架构图}
\end{figure}

\section{系统架构}
为高效协同异构、分布式的边缘计算资源,并满足分布式训练任务对计算与网络资源的联合需求,本系统采用经典的Client-Server架构进行设计。该架构通过中心化的协调与分布式的执行相结合,实现了资源管理的统一性与任务执行的可扩展性。系统主要由一个中心化的主节点(Master) 和多个分布式的边缘节点(Edge Node) 构成。其中,Master节点作为系统的大脑,负责全局资源的管理、作业解析与全局任务调度;而Edge Node作为系统的四肢,负责接收调度指令,并通过容器化技术提供隔离、可控的执行环境来具体完成任务。整体架构的核心目标是在动态、异构的边缘环境中,实现计算与网络资源的协同分配与高效调度。CES系统的具体架构如图\ref{system_framework}所示。


\subsection{Master节点设计}
Master节点是系统的控制中心,其设计着重于全局状态管理、决策制定以及与客户端的交互。它包含以下关键组件,各组件通过内部接口进行高效协作:
\begin{itemize}
\item \textbf{管理器(Manager)}:该组件作为系统对外的首要接口,承担作业的接入与解析职责。它接收用户提交的分布式训练作业描述,解析其中的任务、通信拓扑及资源需求。解析完成后,Manager将结构化的作业信息传递给调度器(Scheduler),并负责将作业状态、执行结果等反馈给用户,管理作业的生命周期。
\item \textbf{控制器(Controller)}:Controller是资源管理的执行层,包含两个专用于处理不同资源维度的子控制器:
\begin{itemize}
    \item \textbf{计算资源控制器(Compute Controller)}:该控制器基于Docker容器技术实现对底层异构计算资源的统一虚拟化抽象与管理。它负责根据调度器的指令,在指定的边缘节点上创建、启动、暂停或销毁Worker容器,从而为每个任务提供计算隔离的运行环境。同时,Compute Controller持续从各边缘节点收集细粒度的资源监控数据(包括CPU利用率、内存占用等),不仅为Scheduler的决策提供实时、准确的依据,也确保了资源分配的公平性和隔离性。
    \item \textbf{网络资源控制器(Network Controller)}:为满足分布式训练中频繁的参数同步与数据交换对网络性能的严格要求,该控制器负责管理边缘节点间的网络资源。其功能涵盖带宽的分配、数据流的路由策略制定以及跨节点的流量转发控制。通过与计算资源的协同管理,Network Controller旨在减少通信瓶颈,保障作业的网络服务质量(QoS)。
\end{itemize}
\item \textbf{调度器(Scheduler)}:作为Master节点的核心决策组件,Scheduler负责协调Compute Controller与Network Controller,实施跨资源维度的联合调度。它持续收集来自各控制器的系统全局信息,包括但不限于所有边缘设备的实时资源利用率、网络链路状态与拓扑、以及排队等待作业的详细配置。基于这些信息,Scheduler运行内嵌的任务调度算法,生成最优的作业调度策略与资源分配方案。其调度决策直接决定了任务被派往哪个边缘节点、获得多少资源以及使用怎样的网络路径,最终目标是优化系统整体的负载均衡度等关键性能指标。
\end{itemize}

\subsection{Edge Node节点设计}
Edge Node是部署在边缘侧的物理或虚拟设备,负责提供具体的计算和网络能力。每个Edge Node在设计上需具备自治的资源管理能力和与Master协同工作的通信能力,其内部组件如下:
\begin{itemize}
\item \textbf{通信器(Messenger)}:该组件是Edge Node与外界通信的枢纽。它一方面与Master节点保持持久的心跳连接,定期上报本节点的资源状态(通过Manager收集),并接收来自Master的调度指令与控制命令(如创建Worker);另一方面,在需要节点间直接通信的任务中(如分布式训练中的参数服务器与Worker之间),Messenger也负责与其他Edge Node建立点对点的数据通信通道,高效传输中间数据或模型参数。
\item \textbf{管理器(Manager)}:每个Edge Node拥有一个本地的管理器,它是Master端Controller在边缘侧的代理和执行延伸,也包含两个部分:
\begin{itemize}
    \item \textbf{计算资源管理器(Compute Manager)}:负责管理本节点所有计算资源的具体分配。它接收并执行来自Master Compute Controller的容器操作指令,通过调用本地的Docker来实际创建和管理Worker容器。同时,它持续监控本节点各容器的资源消耗情况,并将聚合后的数据通过Messenger上报。
    \item \textbf{网络资源管理器(Network Manager)}:负责执行Master的Network Controller下发的网络策略。它在本地通过流量控制工具Traffic Control来实施具体的带宽限制、流量整形或路由规则,确保节点出/入口的网络流量符合全局调度方案的要求。
\end{itemize}
\item \textbf{工作器(Worker)}:Worker是由Docker容器实例化的任务执行单元,是实际执行用户作业中具体任务的实体。每个Worker运行在资源隔离的容器环境中,内部包含用户指定的训练框架、应用程序代码及依赖库。Worker根据作业逻辑进行本地计算,并通过Messenger组件与其他Worker或参数服务器进行通信,共同完成分布式训练任务。
\end{itemize}

综上所述,本系统通过Master节点的集中式智能调度与Edge Node节点的分布式高效执行相结合,构建了一个层次化、松耦合的协同计算框架。各组件各司其职又紧密联动,为在复杂边缘环境下开展资源敏感的分布式训练任务提供了坚实的系统基础。

\section{任务调度流程}
该系统的任务调度流程遵循从作业描述到资源分配、最终至任务执行的清晰逻辑过程,旨在实现跨资源维度的协同优化。整个过程可划分为三个阶段,其详细流程如下:

首先,在作业提交与解析阶段,用户通过客户端向Master节点的Manager组件提交分布式训练作业。该作业通常以资源描述文件进行定义,其中完整定义了作业的规格,包括:任务类型、待执行的程序实体(如镜像名称或脚本路径)、计算资源需求(例如,所需的CPU核数及内存大小)、任务间的网络带宽需求及通信拓扑。Manager接收并解析此作业描述,将其转化为系统内部可识别的结构化任务元数据,为后续的智能调度提供精确的决策依据。

随后,进入资源联合调度与决策阶段,此为系统的核心环节。Master中的Scheduler(即联合调度器)被触发,它综合多源信息进行全局决策:
\begin{itemize}
\item \textbf{任务侧信息}:来自Manager的作业元数据,特别是任务资源需求与通信拓扑。
\item \textbf{系统侧信息}:Controller持续收集并维护的全局资源视图,包括各Edge Node的实时计算资源利用率、可用网络带宽以及链路拓扑。
\end{itemize}

基于上述信息,Scheduler运行其内嵌的协同调度算法。该算法旨在平衡边缘节点的负载、最小化通信开销并满足资源约束,最终生成一个细粒度的调度策略。此策略具体规定了:1) 每个任务实例被分配到的目标Edge Node;2) 为任务间关键数据流分配的带宽配额;3) 高效的数据传输路径或路由方案。此策略体现了系统“数据与计算协同感知”的核心设计思想。

最后,在任务分发、执行与监控阶段,调度策略被下发并执行。Master的Controller将任务启动指令及资源配置要求下发至目标Edge Node。各Edge Node的本地Manager协同工作:Compute Manager根据指令通过Docker创建指定规格的Worker容器;Network Manager则配置本地的流量控制规则以实施分配的带宽方案。任务开始在隔离的容器中执行,期间所有Worker的状态(如运行、完成、失败)及各节点的资源消耗数据通过Messenger组件实时反馈至Master的Controller。这些动态信息构成了一个闭环反馈,使得Scheduler能够进行潜在的动态任务迁移或资源再分配,以应对负载波动或节点故障,从而保障系统的整体鲁棒性与执行效率。

综上所述,该工作流程通过“解析-决策-执行-反馈”的过程,实现了对边缘异构资源的精细化管理与自适应调度,确保了分布式训练作业在复杂边缘环境中的高效、可靠运行。

\section{本章小结}
本章详细阐述了协同边缘计算任务调度系统CES的设计与实现。首先明确了系统面向弹性环境中具有通信依赖的分布式训练任务,确立了计算与网络资源联合调度的核心设计目标,具体包括提升真实环境调度可信度、资源利用均衡度与作业性能隔离。随后,系统采用主从架构,详细设计了中心化Master节点与分布式Edge Node节点的内部结构及交互机制,实现了全局资源统一视图与协同决策。最后,系统描述了从作业提交、资源联合调度到任务分发执行的完整调度流程,通过“解析-决策-执行-反馈”闭环机制保障了动态环境下的调度鲁棒性。本章为后续算法设计与实验验证提供了完整的系统支撑。%第三章
	\chapter{基于深度强化学习的联合任务调度算法}
\section{问题建模}
\section{算法设计}
\section{实验结果与分析}%第四章
	\chapter{多任务队列调度算法}
针对协同边缘计算环境中多分布式学习作业的在线调度问题,本文提出一种融合滚动窗口、资源感知贪心排序与动态饥饿保护的混合启发式调度算法。该算法在CPU、内存、链路带宽等多维资源约束下,联合优化带宽满足度、系统负载均衡度与作业平均等待时间三个相互冲突的指标。首先,基于系统负载动态确定候选作业窗口,平衡决策复杂度与调度机会;其次,通过资源适配度、业务优先级与等待因子计算综合评分,优先调度与剩余资源最匹配的作业;最后,引入跳过计数阈值与强制预留机制防止低优先级作业饥饿。本文详细给出了问题形式化模型、算法伪代码与复杂度分析,并设计了完整的仿真实验方案,包括物理拓扑生成、作业到达过程、对比算法与多维度评估指标,为后续实现与性能验证奠定理论基础。
\section{问题建模}
随着深度学习模型在边缘智能场景中的广泛应用,越来越多的分布式学习作业需要部署在资源受限的边缘节点上。这些作业通常包含多个协同计算的任务节点及任务间的通信链路,对计算资源(CPU、内存)与网络资源(带宽)均有需求。如何在满足物理资源约束的前提下,高效、公平地调度多批作业,已成为边缘计算资源管理领域的关键挑战。

现有研究往往将作业调度分解为两个独立阶段:先决定作业的执行顺序,再为单个作业分配计算节点与带宽资源。这种分阶段方法容易造成目标冲突——例如,仅按优先级排序可能忽略资源匹配度,导致资源碎片化;仅按资源需求大小排序则可能使大作业长期阻塞小作业。此外,多数调度器未显式考虑作业间的带宽竞争,或仅以最小带宽保障为目标,难以逼近最大带宽需求。

为解决上述问题,本文提出一种\textbf{多作业队列在线调度算法}(Multi-Job Online Scheduling, MJOS),其核心创新包括:

\begin{itemize}
    \item \textbf{联合调度框架}:将计算资源与网络资源统一纳入调度决策,避免分阶段优化的次优性;
    \item \textbf{动态滚动窗口}:根据当前运行队列的负载情况自适应调整候选作业窗口大小,在决策开销与调度性能间取得平衡;
    \item \textbf{资源感知评分}:从CPU、内存、带宽三个维度计算作业需求与系统剩余资源的匹配程度,优先调度最能填补资源空缺的作业;
    \item \textbf{饥饿保护机制}:通过跳过计数与等待时间监控,对长期未获调度的作业实施优先级提升或强制资源预留,保障公平性。
\end{itemize}

本文所述算法与现有的单作业调度器(如基于强化学习的PPO分配器)完全兼容,可作为上层调度器集成至CES(Collaborative Edge System)等边缘计算平台。下文将详细介绍问题建模、算法设计及复杂度分析。

\subsection{系统模型}

考虑一个协同边缘计算系统,包含一个中心调度器及若干边缘节点。系统维护两个队列:\textbf{等待调度队列} $\mathcal{Q}_{\text{wait}}$ 存放已到达但尚未获得资源的作业;\textbf{运行队列} $\mathcal{Q}_{\text{run}}$ 存放正在执行的作业。调度器在每个离散时间步 $t$ 观察系统状态,从 $\mathcal{Q}_{\text{wait}}$ 中选择若干作业尝试分配资源,成功则移入 $\mathcal{Q}_{\text{run}}$。

\subsubsection{物理网络资源}
\begin{itemize}
    \item 物理节点集合:$\mathcal{P} = \{p_1, p_2, \dots, p_M\}$,$M = |\mathcal{P}|$。
    \item 物理链路集合:$\mathcal{R} = \{r_1, r_2, \dots, r_L\}$,$L = |\mathcal{R}|$。链路为双向,带宽对称。
    \item 节点 $p$ 的总资源:$C_p^{\text{CPU}}$(CPU核数或单位),$R_p^{\text{RAM}}$(内存大小)。运行中占用资源为已分配给运行作业任务的部分,剩余空闲资源记为 $A_p^{\text{CPU}}(t)$、$A_p^{\text{RAM}}(t)$。
    \item 链路 $r$ 的总带宽:$B_r^{\text{total}}$。已分配带宽记为 $U_r(t)$,剩余带宽 $A_r^{\text{BW}}(t) = B_r^{\text{total}} - U_r(t)$。
\end{itemize}

\subsubsection{作业模型}

每个作业 $j \in \mathcal{J}$ 具有以下属性:
\begin{itemize}
    \item 任务节点集 $\mathcal{N}_j = \{n_{j1}, \dots, n_{jK_j}\}$,$K_j = |\mathcal{N}_j|$。
    \item 通信链路集 $\mathcal{V}_j = \{v_{j1}, \dots, v_{jE_j}\}$,$E_j = |\mathcal{V}_j|$,每条链路连接两个任务节点。
    \item 任务节点 $n$ 的资源需求:$\text{CPU}^N(n)$,$\text{RAM}^N(n)$。实际映射时,每个任务需独占式分配完整需求。
    \item 通信链路 $v$ 的带宽需求区间:$[b_v^{\min}, b_v^{\max}]$,调度器应至少分配 $b_v^{\min}$,并尽可能接近 $b_v^{\max}$。
    \item 优先级权重 $w_j \in [1,5]$,整数,数值越大表示业务紧急程度越高。
    \item 到达时间 $t_j^{\text{arr}}$;开始时间 $t_j^{\text{start}}$;等待时间 $T_j^{\text{wait}} = t_j^{\text{start}} - t_j^{\text{arr}}$(若尚未开始,则为 $t - t_j^{\text{arr}}$)。
\end{itemize}

\subsubsection{映射与资源分配约束}

设二元决策变量 $X_{p}^{n}(j) \in \{0,1\}$ 表示作业 $j$ 的任务 $n$ 是否放置在物理节点 $p$;$Y_{r}^{v}(j) \in \{0,1\}$ 表示作业 $j$ 的虚拟链路 $v$ 是否占用物理链路 $r$。则约束如下:

\begin{enumerate}
    \item \textbf{节点资源约束}(任一时刻 $t$):
    \begin{align}
        &\forall p \in \mathcal{P}: \sum_{j \in \mathcal{Q}_{\text{run}}(t)} \sum_{n \in \mathcal{N}_j} X_{p}^{n}(j) \cdot \text{CPU}^N(n) \le C_p^{\text{CPU}}, \\
        &\forall p \in \mathcal{P}: \sum_{j \in \mathcal{Q}_{\text{run}}(t)} \sum_{n \in \mathcal{N}_j} X_{p}^{n}(j) \cdot \text{RAM}^N(n) \le R_p^{\text{RAM}}.
    \end{align}
    
    \item \textbf{链路带宽约束}(任一时刻 $t$):
    \begin{equation}
        \forall r \in \mathcal{R}: \sum_{j \in \mathcal{Q}_{\text{run}}(t)} \sum_{v \in \mathcal{V}_j} Y_{r}^{v}(j) \cdot b_v^{\text{alloc}} \le B_r^{\text{total}},
    \end{equation}
    其中 $b_v^{\text{alloc}} \in [b_v^{\min}, b_v^{\max}]$ 为实际分配给链路 $v$ 的带宽。
    
    \item \textbf{任务映射唯一性}:
    \begin{equation}
        \forall j, \forall n \in \mathcal{N}_j: \sum_{p \in \mathcal{P}} X_{p}^{n}(j) = 1.
    \end{equation}
    
    \item \textbf{链路路径约束}:若任务 $n_i$ 映射至 $p_a$,任务 $n_k$ 映射至 $p_b$,且虚拟链路 $v = (n_i, n_k)$,则 $Y_{r}^{v}(j)=1$ 当且仅当物理链路 $r$ 位于从 $p_a$ 到 $p_b$ 的最短路径上。本文假设采用静态最短路径路由。
    
    \item \textbf{作业非抢占}:作业一旦开始执行,在完成前不中断,不释放资源。
\end{enumerate}

\subsection{性能指标}

为量化调度质量,定义三个核心指标,分别对应带宽、负载均衡与延迟。

\subsubsection{带宽需求满足度}

对于运行中的作业 $j$,定义其带宽满足度为所有虚拟链路实际分配带宽与最大需求比值的平均值:
\begin{equation}
    D_{\text{BW}}(j) = \frac{1}{|\mathcal{V}_j|} \sum_{v \in \mathcal{V}_j} \frac{b_v^{\text{alloc}}}{b_v^{\max}}.
\end{equation}
若作业所有任务映射至同一节点(无通信),则定义 $D_{\text{BW}}(j) = 1$。系统整体带宽满足度为运行队列中作业的平均值:
\begin{equation}
    \bar{D}_{\text{BW}}(t) = \frac{1}{|\mathcal{Q}_{\text{run}}(t)|} \sum_{j \in \mathcal{Q}_{\text{run}}(t)} D_{\text{BW}}(j).
\end{equation}

\subsubsection{系统负载均衡度}

负载均衡度衡量三类资源在物理节点/链路上的利用分布均匀性。首先计算节点 $p$ 的CPU利用率 $\rho_p^{\text{CPU}}(t) = 1 - A_p^{\text{CPU}}(t)/C_p^{\text{CPU}}$,内存利用率 $\rho_p^{\text{RAM}}(t)$ 类似。链路 $r$ 的带宽利用率 $\rho_r^{\text{BW}}(t) = U_r(t)/B_r^{\text{total}}$。

定义资源 $x$(CPU、RAM、BW)的\textbf{不均衡系数}为各实体利用率的变异系数(标准差/均值):
\begin{equation}
    L_t^{x} = \frac{\sqrt{\frac{1}{M_x} \sum_{i=1}^{M_x} (\rho_i^x - \bar{\rho}^x)^2}}{\bar{\rho}^x + \epsilon},
\end{equation}
其中 $M_x$ 为对应实体数量(节点数或链路数),$\epsilon$ 避免除零。整体负载均衡度加权组合:
\begin{equation}
    L_t = w_1 L_t^{\text{CPU}} + w_2 L_t^{\text{RAM}} + w_3 L_t^{\text{BW}}, \quad w_1+w_2+w_3=1.
\end{equation}
本文取默认权重 $(0.4,0.4,0.2)$。$L_t$ 越小表示资源利用越均衡。

\subsubsection{作业平均等待时间}

在时刻 $t$,所有\textbf{已到达}作业的平均等待时间:
\begin{equation}
    \bar{T}_{\text{wait}}(t) = \frac{1}{|\mathcal{Q}_{\text{wait}}(t) \cup \mathcal{Q}_{\text{run}}(t)|} \sum_{j \in \mathcal{Q}_{\text{wait}} \cup \mathcal{Q}_{\text{run}}} (t - t_j^{\text{arr}}).
\end{equation}
对于已完成作业,其等待时间固定为 $t_j^{\text{start}}-t_j^{\text{arr}}$。

\subsection{优化目标}

本文采用\textbf{加权和法}将多目标问题转化为单目标优化。在每个调度时刻 $t$,选择待调度作业子集 $S_t \subseteq \mathcal{Q}_{\text{wait}}(t)$,使得调度后的系统综合性能最大:
\begin{equation}
    \max_{S_t} \; \alpha_1 \bar{D}_{\text{BW}}(t+|S_t|) + \alpha_2 \left(1 - L_{t+|S_t|}\right) + \alpha_3 \left(1 - \frac{\bar{T}_{\text{wait}}(t+|S_t|)}{T_{\max}}\right),
\end{equation}
其中 $\alpha_1=0.5$,$\alpha_2=0.3$,$\alpha_3=0.2$,$T_{\max}$ 为最大容忍等待时间(设为仿真长度或经验值)。约束条件同第\ref{sec:model}节。

由于未来状态依赖于当前调度决策且作业到达随机,精确在线优化极为困难。因此,本文设计一种启发式算法,在每个时刻仅调度\textbf{至多一个}作业(或少量作业),以贪心方式逼近上述目标。

\section{算法设计}
\subsection{总体框架}
本文提出的\textbf{多作业在线调度算法}(Multi-Job Online Scheduling, MJOS)由四个核心模块构成:

\begin{enumerate}
    \item \textbf{滚动窗口选择}:根据当前系统负载,从 $\mathcal{Q}_{\text{wait}}$ 中选出有限个作业作为候选集 $C$,避免每步遍历整个队列。
    \item \textbf{综合评分计算}:对每个候选作业,计算资源适配度、优先级得分与等待因子,加权得到总评分。
    \item \textbf{贪心调度尝试}:按评分降序依次调用单作业调度器尝试分配资源,成功则更新系统状态并移动作业。
    \item \textbf{饥饿保护}:监控长期未获调度的作业,通过提升优先级或强制预留方式提高其调度机会。
\end{enumerate}

算法\ref{alg:MJOS}给出了MJOS的顶层伪代码。

\begin{algorithm}[htbp]
    \caption{多作业在线调度算法(MJOS)}
    \label{alg:MJOS}
    \begin{algorithmic}[1]
        \Require 等待队列 $\mathcal{Q}_{\text{wait}}$,运行队列 $\mathcal{Q}_{\text{run}}$,物理网络状态 $(\mathcal{P},\mathcal{R})$,参数 $\beta,\tau,\theta,K,T_{\text{th}}$
        \Ensure 更新后的 $\mathcal{Q}_{\text{wait}}$,$\mathcal{Q}_{\text{run}}$ 及资源状态
        
        \For{每个调度周期}
            \State // 步骤1:滚动窗口选择
            \State $W \gets \min\left( \left\lfloor \beta \cdot \dfrac{\sum_{p} A_p^{\text{CPU}}}{ \max_p C_p^{\text{CPU}} } \cdot |\mathcal{P}| \right\rfloor,\; |\mathcal{Q}_{\text{wait}}| \right)$
            \State $C \gets \{j_1, j_2, \dots, j_W\}$,即 $\mathcal{Q}_{\text{wait}}$ 的前 $W$ 个作业(按到达时间排序)
            
            \State // 步骤2:为每个候选作业计算综合评分
            \For{each $j \in C$}
                \State $R_{\text{match}}(j) \gets \text{ComputeResourceMatch}(j)$  \Comment{式(13)}
                \State $S_{\text{wait}}(j) \gets 1 - \exp\left(-\dfrac{t - t_j^{\text{arr}}}{\tau}\right)$
                \State $S_{\text{total}}(j) \gets 0.6 \cdot R_{\text{match}}(j) + 0.3 \cdot \dfrac{w_j}{5} + 0.1 \cdot S_{\text{wait}}(j)$
            \EndFor
            
            \State // 步骤3:按评分降序尝试调度
            \State 将 $C$ 按 $S_{\text{total}}$ 降序排列
            \For{each $j$ in sorted $C$}
                \State $success \gets \text{TryMapJob}(j)$  \Comment{调用单作业调度器}
                \If{$success$}
                    \State 将 $j$ 从 $\mathcal{Q}_{\text{wait}}$ 移至 $\mathcal{Q}_{\text{run}}$,更新资源状态
                    \State 重置 $j$ 的跳过计数 $\text{skip}(j) \gets 0$
                    \State \textbf{break}  \Comment{本周期只调度一个作业(可根据配置调度多个)}
                \Else
                    \State $\text{skip}(j) \gets \text{skip}(j) + 1$
                \EndIf
            \EndFor
            
            \State // 步骤4:饥饿保护处理
            \For{each $j \in \mathcal{Q}_{\text{wait}}$}
                \If{$t - t_j^{\text{arr}} > T_{\text{th}}$}
                    \State 临时提升 $w_j \gets \min(w_j+1, 5)$  \Comment{优先级提升}
                \EndIf
                \If{$\text{skip}(j) \ge K$}
                    \State $\text{TryForceMap}(j)$  \Comment{尝试预留最小资源包}
                \EndIf
            \EndFor
        \EndFor
    \end{algorithmic}
\end{algorithm}

\subsection{滚动窗口机制}
窗口大小 $W_t$ 动态计算,其基本思想是:当系统剩余资源较多时,应扩大候选窗口,增加调度机会;当系统资源紧张时,缩小窗口,避免频繁失败尝试。具体公式:
\begin{equation}
    W_t = \min\left( \left\lfloor \beta \cdot \frac{\sum_{p \in \mathcal{P}} A_p^{\text{CPU}}(t)}{\max_{p \in \mathcal{P}} C_p^{\text{CPU}}} \cdot |\mathcal{P}| \right\rfloor,\; |\mathcal{Q}_{\text{wait}}(t)| \right),
\end{equation}
其中 $\beta \in [1,3]$ 为窗口扩展因子,默认 $\beta=2$。分母使用最大节点CPU容量以归一化。若全部节点CPU空闲率均为100\%,则 $W_t = \beta |\mathcal{P}|$;若系统满载,则 $W_t=0$,不调度新作业。窗口内作业按\textbf{先入先出}顺序选取,即最早到达的 $W_t$ 个作业。

该机制与实现代码中 $W = \min(\text{max\_load\_factor} \cdot |P|, |\text{queue}|)$ 形式兼容,但此处将固定因子 $\text{max\_load\_factor}$ 替换为负载自适应因子,更符合理论模型的动态特性。

\subsection{资源适配度计算}
资源适配度 $R_{\text{match}}(j)$ 衡量作业 $j$ 的总资源需求与当前网络剩余资源的匹配程度。为简化计算,采用\textbf{最大空闲资源归一化}方法:分别找出CPU、内存、带宽在各自实体中的最大空闲量,然后计算需求与该最大值的比值,并截断至1。数学表达式为:
\begin{equation}
    R_{\text{match}}(j) = \frac{1}{3} \left[ \min\left(1, \frac{\text{CPU}^{\text{dem}}(j)}{\max_{p \in \mathcal{P}} A_p^{\text{CPU}}}\right) + \min\left(1, \frac{\text{RAM}^{\text{dem}}(j)}{\max_{p \in \mathcal{P}} A_p^{\text{RAM}}}\right) + \min\left(1, \frac{\text{BW}^{\text{dem}}(j)}{\max_{r \in \mathcal{R}} A_r^{\text{BW}}}\right) \right],
    \label{eq:match}
\end{equation}
其中:
\begin{itemize}
    \item $\text{CPU}^{\text{dem}}(j) = \sum_{n \in \mathcal{N}_j} \text{CPU}^N(n)$,作业总CPU需求;
    \item $\text{RAM}^{\text{dem}}(j) = \sum_{n \in \mathcal{N}_j} \text{RAM}^N(n)$,作业总内存需求;
    \item $\text{BW}^{\text{dem}}(j) = \sum_{v \in \mathcal{V}_j} b_v^{\min}$(或使用 $b_v^{\max}$,本文采用 $b_v^{\min}$ 作为保守估计)。
\end{itemize}

该归一化方式计算简单,且能反映作业需求是否超过当前网络中任何单节点/链路的剩余能力。若所有空闲资源均大于需求,则 $R_{\text{match}}=1$;若需求超过最大空闲资源,则比值小于1,鼓励调度需求较小的作业。

\subsection{综合评分}
每个候选作业的综合评分 $S_{\text{total}}(j)$ 由三部分加权组成:

\begin{itemize}
    \item \textbf{资源适配度} $R_{\text{match}}(j)$,权重0.6 —— 反映即时资源匹配程度,是调度决策的主要依据。
    \item \textbf{优先级得分} $\dfrac{w_j}{5}$,权重0.3 —— 归一化至 $[0.2,1]$,体现业务紧急程度。
    \item \textbf{等待时间因子} $S_{\text{wait}}(j)$,权重0.1 —— 采用指数函数 $S_{\text{wait}}(j)=1-\exp\left(-\dfrac{t-t_j^{\text{arr}}}{\tau}\right)$,其中 $\tau$ 为等待敏感参数(默认 $\tau=50$)。等待时间越长,该因子越接近1,防止作业饥饿。
\end{itemize}

因此:
\begin{equation}
    S_{\text{total}}(j) = 0.6 \cdot R_{\text{match}}(j) + 0.3 \cdot \frac{w_j}{5} + 0.1 \cdot S_{\text{wait}}(j).
    \label{eq:score}
\end{equation}

\subsection{饥饿保护策略}

为保障长期公平性,MJOS引入双重饥饿防护:

\begin{enumerate}
    \item \textbf{等待时间阈值提升}:若作业等待时间超过 $T_{\text{th}}$(例如100个时间单位),则将其优先级临时提升1级(不超过最大值5),使其在后续评分中获得更高优先级权重。
    \item \textbf{强制预留调度}:若作业连续被跳过 $K$ 次(例如 $K=3$),则触发强制调度尝试:系统为该作业预留一份“最小资源包”——例如,为每个任务预留最小需求(CPU/RAM),并为通信链路预留 $b_v^{\min}$ 带宽。若预留成功,则立即调度该作业;否则保留其跳过计数,待后续资源充足时再尝试。
\end{enumerate}

这两种保护机制协同工作:前者提高评分排名,后者通过资源预留保证极端情况下的调度机会。

\subsection{算法复杂度分析}

每调度周期的主要开销在于:
\begin{itemize}
    \item 计算 $W$ 个作业的综合评分:$O(W)$。
    \item 排序 $W$ 个作业:$O(W \log W)$。
    \item 每个作业尝试映射:单作业调度器复杂度记为 $O(f(K_j, M, L))$。典型贪心映射为 $O(K_j M + \text{SPF cost})$,其中SPF为最短路径计算,可采用Floyd-Warshall预处理后 $O(1)$ 查询,因此总开销 $O(K_j M)$。
    \item 饥饿保护遍历 $\mathcal{Q}_{\text{wait}}$:$O(|\mathcal{Q}_{\text{wait}}|)$。
\end{itemize}

实际中 $W$ 与 $|\mathcal{Q}_{\text{wait}}|$ 均受限于系统规模与作业到达率,算法可在毫秒级完成,适用于在线调度。

\section{实验设计}
\label{sec:experiment}

为验证MJOS算法的有效性与鲁棒性,本节设计一套完整的仿真实验框架,涵盖物理网络生成、作业负载生成、对比算法、评估指标及统计分析方法。

\subsection{仿真环境设置}

\subsubsection{物理网络拓扑}
\begin{itemize}
    \item 节点数 $M \in \{10,15,20\}$。
    \item 链路生成:随机连通图,任意节点对连接概率 $p_{\text{edge}}=0.3$,并保证图连通;若连通分量多于1,则随机添加边直至全连通。
    \item 节点资源:CPU与内存容量在区间 $[50,100]$ 内均匀随机采样,单位为“单位”。
    \item 链路带宽:在 $[100,1000]$ Mbps内均匀随机采样。
\end{itemize}

\subsubsection{作业到达过程}
\begin{itemize}
    \item 作业到达间隔服从泊松分布,到达率 $\lambda \in \{2,3,5\}$(单位:作业/时间单位)。
    \item 每个作业的任务节点数 $K_j \sim \text{Uniform}\{3,4,5,6,7,8\}$。
    \item 通信拓扑:在 $K_j$ 个任务节点间以概率 $p_{\text{comm}}=0.4$ 添加虚拟链路,并保证子图连通。
    \item 任务CPU需求:$\text{CPU}^N(n) \sim \text{Uniform}[10,50]$。
    \item 任务内存需求:$\text{RAM}^N(n) \sim \text{Uniform}[10,50]$。
    \item 带宽需求区间:每条虚拟链路的 $b_v^{\min} \sim \text{Uniform}[10,50]$,$b_v^{\max} = b_v^{\min} + \text{Uniform}[0,50]$,保证 $b_v^{\max} \ge b_v^{\min}$。
    \item 优先级权重 $w_j \sim \text{Uniform}\{1,2,3,4,5\}$。
    \item 仿真总时长:$T_{\text{sim}} = 1000$ 时间单位。
    \item 作业总数:动态生成,约为 $\lambda \cdot T_{\text{sim}}$,约200~500个。
\end{itemize}

\subsubsection{系统初始负载}
为考察不同资源紧张程度下的算法性能,设置三种初始占用水平:
\begin{itemize}
    \item \textbf{宽松}:随机将节点资源的30\%预分配给若干模拟作业(运行队列中);
    \item \textbf{中等}:50\%初始占用;
    \item \textbf{紧张}:70\%初始占用。
\end{itemize}

\subsection{对比算法}

选取五种代表性调度策略作为基准。


所有算法均使用相同的单作业映射器(默认贪心首次适应)以保证公平对比。后续可扩展实验对比不同单作业调度器的影响。



所有指标均在仿真结束时统计,部分时间序列指标(如平均等待时间、带宽满足度)按每100个时间单位采样记录演化趋势。

\subsection{实验方案与统计方法}




\section{本章小结}
本文面向协同边缘计算环境下的多分布式学习作业调度问题,提出了一种在线启发式调度算法MJOS。该算法通过动态滚动窗口控制决策规模,利用资源感知评分引导作业调度顺序,并引入等待时间阈值与强制预留机制防止饥饿,综合优化了带宽满足度、负载均衡度与平均等待时间三个相互冲突的指标。与现有分阶段调度方法相比,MJOS将计算与网络资源联合考虑,且与任意单作业调度器兼容,具有良好的可扩展性与实用性。

文章完成了问题形式化建模、算法详细设计及完整的仿真实验方案。后续工作将基于所述实验设计进行代码实现与大规模仿真验证,并探索以下方向:(1)将窗口自适应因子 $\beta$ 与等待敏感参数 $\tau$ 动态在线调优;(2)引入多作业并行调度能力,进一步提升资源利用率;(3)将MJOS扩展至异构资源(如GPU、NPU)场景。
%第五章

	\backmatter %章节不编号但页码继续
	%%%%%%%%%%%%%%%%%%%%%%%%%%%%%%%%%%%%%%%%%%%%%%%%%%%%%%%%%%%%%%    微调,使得后续章节的页眉不带章号
	\renewcommand{\chaptermark}[1]{\markboth{#1}{}}
	%%%%%%%%%%%%%%%%%%%%%%%%%%%%%%%%%%%%%%%%%%%%%%%%%%%%%%%%%%%%%%
	\chapter{总结与展望}
文章完成了问题形式化建模、算法详细设计及完整的仿真实验方案。后续工作将探索以下方向:(1)将窗口自适应因子 $\beta$ 与等待敏感参数 $\tau$ 动态在线调优;(2)引入多作业并行调度能力,进一步提升资源利用率;(3)将CES-Multi-Job扩展至异构资源(如GPU、NPU)场景。 %结论
	 %%%%%%%%%%%%%%%%%%%%%%%%%%%%%%%%%%%%%%%%%%%%%% bibtex参考文献设置  (原版)
%%	\bibliographystyle{scutthesis}
%%	\bibliography{F:/MyLibrary}
	%%%%%%%%%%%%%%%%%%%%%%%%%%%%%%%%%%%%%%%%%%%%%%
	%%%%%%%%%%%%%%%%%%%%%%%%%%%%%%%%%%%%%%%%%%%%%% biber参考文献设置	
	%\renewcommand*{\bibfont}{\refbodyfont}			% 设置文献著录字号比正文小一号(五号),需要小四号请注释该行. % 不推荐使用small,而是使用cls文件中精确定义了的字号。
	\phantomsection % “目录”中的链接能正确跳转,需要添加 \phantomsection 否则点击参考文献会跳转到结论
	\addcontentsline{toc}{chapter}{参考文献}	%目录中添加参考文献
	\printbibliography	% 参考文献著录
 	%%%%%%%%%%%%%%%%%%%%%%%%%%%%%%%%%%%%%%%%%%%%%%
 	% 只有一个附录
% 	\include{chapter/appendix}
 	% 有多个附录
	% \include{chapter/appendix1} %附录1
	% \include{chapter/appendix2} %附录2
 	%%%%%%%%%%%%%%%%%%%
	\include{chapter/pub} %成果
	\chapter{致\texorpdfstring{\quad}{}谢}
时光荏苒,三年的研究生学习生涯即将落下帷幕。回首这段宝贵的历程,我在不断探索与磨砺中收获了成长与沉淀。在此,谨向所有曾给予我关怀与支持的师长、同窗和亲人致以最诚挚的谢意。

首先,我要由衷地感谢我的导师。他不仅在学术上给予我悉心的指导,更以开阔的视野和严谨的治学态度,深深影响了我对科研的理解。从选题初探、问题梳理,到方法构建与论文撰写,每一步都离不开他耐心的点拨和深刻的建议。面对困难时,他的鼓励与信任让我一次次重拾信心、坚定方向。他执着探索、勇于突破的精神,始终是我学习的榜样。

同时,我也要感谢软件工程专业的各位老师和同窗好友。老师们的言传身教,使我对专业知识的理解更加系统深入,也让我体会到科研的严谨与魅力。在遇到生活与科研瓶颈时,同学们真诚的建议与陪伴,使我得以顺利度过一个个难关。彼此的鼓励与陪伴,让这三年充满了温情与动力。

我还要深深感谢我的家人与朋友。他们始终是我最坚实的后盾,分享我的喜悦,抚慰我的低落,是他们的理解与守护,让我在前行的路上从不孤单。

最后,感谢参与本次论文评审与答辩的各位专家老师,感谢你们在百忙之中给予我的指导和宝贵建议。你们的意见让我的工作更加完善,也让我对未来的研究之路充满敬畏与期待。 %致谢
\end{document}
