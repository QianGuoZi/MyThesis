\chapter{摘\texorpdfstring{\quad}{}要}
	协同边缘计算作为一种新兴范式,通过利用地理分布且异构的边缘节点资源,为自动驾驶、工业互联网等低延迟、高可靠应用提供了技术支撑。然而,边缘环境的异构性、计算与网络资源的紧密耦合性,以及分布式智能应用(如分布式训练)所呈现的复杂网络依赖拓扑,使得传统的任务调度机制在资源效率和系统性能上面临严峻挑战。当前研究主要存在两大局限:其一,主流基于有向无环图的任务依赖模型难以有效刻画可能具有环状等复杂通信特征的分布式智能任务;其二,现有调度方案普遍将计算资源与网络资源进行独立管理与调度,忽略了二者间的内在权衡关系,容易导致资源竞争与性能瓶颈。为解决上述问题,本文在协同边缘计算环境下设计了一种高效的计算与网络资源联合调度方案,主要工作如下:

	本文设计并实现了协同边缘调度系统 CES(Collaborative Edge Scheduler),首次将任务到节点的映射与网络带宽分配进行联合管理。其次,构建了一个计算与网络资源的联合优化模型,该模型能够精确描述分布式智能应用的通信拓扑与资源需求,并弹性适应动态环境。我们进一步提出了一种基于深度强化学习的智能调度算法,采用双分支演员-评论家架构,并融入启发式奖励函数,使系统能够在动态环境中自主学习全局最优的联合调度策略,从而实现系统负载均衡与用户带宽满足度等整体性能的优化。仿真实验和系统实验结果表明,该算法在整体性能上显著优于现有基准方法。

	此外,为应对多分布式学习作业并发的场景,本文在 CES 基础上进一步提出了 CES-Multi 算法。该算法以提升资源利用均衡度、保障作业带宽需求满足度及降低作业平均等待时间为目标,在边缘节点多资源约束下,融合滚动窗口机制、资源感知贪心排序与动态饥饿保护策略。该算法通过滚动窗口动态选择候选作业,依据资源适配度、业务优先级和等待时间进行综合评分与排序,并调用既有单作业调度器执行资源分配;同时通过优先级动态提升与最小资源预留等机制,避免低优先级作业饥饿。实验表明,CES-Multi 在资源利用率、作业等待时间与带宽满足度方面均优于现有方法,有效增强了 CES 系统在多作业调度场景中的适用性与综合性能。

\keywordsCN{边缘计算;任务调度;多目标优化;深度强化学习;分布式学习任务}

\chapter{Abstract}
	

\keywordsEN{\LaTeX{}; Paper}