\chapter{摘\texorpdfstring{\quad}{}要}
协同边缘计算作为一种新兴范式,通过利用地理分布且异构的边缘节点资源,为自动驾驶、工业互联网等低延迟、高可靠应用提供了技术支撑。然而,边缘环境的异构性、计算与网络资源的紧密耦合性,以及分布式智能应用(如分布式训练)所呈现的复杂网络依赖拓扑,使得传统的任务调度机制在资源效率和系统性能上面临严峻挑战。当前研究主要存在两大局限:其一,主流基于有向无环图的任务依赖模型难以有效刻画可能具有环状等复杂通信特征的分布式智能任务;其二,现有调度方案普遍将计算资源与网络资源进行独立管理与调度,忽略了二者间的内在权衡关系,容易导致资源竞争与性能瓶颈。为解决上述问题,本文在协同边缘计算环境下设计了一种高效的计算与网络资源联合调度方案,主要工作如下:

本文设计并实现了协同边缘调度系统CES(Collaborative Edge Scheduler),首次将任务到节点的映射与网络带宽分配进行联合管理。其次,构建了一个计算与网络资源的联合优化模型,该模型能够精确描述分布式训练等任务的通信拓扑与资源需求,并弹性适应动态环境。我们进一步提出了一种基于深度强化学习的智能调度算法,采用双分支演员-评论家架构,并融入启发式奖励函数,使系统能够在动态环境中自主学习全局最优的联合调度策略,从而实现系统负载均衡与用户带宽满足度等整体性能的优化。仿真实验和系统实验结果表明,该算法在整体性能上显著优于现有基准方法。

此外,为应对多分布式学习作业并发的场景,本文在CES基础上进一步提出了CES-Multi算法。该算法以提升资源利用均衡度、保障作业带宽需求满足度及降低作业平均等待时间为目标,在边缘节点多资源约束下,融合滚动窗口机制、资源感知贪心排序与动态饥饿保护策略。该算法通过滚动窗口动态选择候选作业,依据资源适配度、业务优先级和等待时间进行综合评分与排序,并调用既有单作业调度器执行资源分配;同时通过优先级动态提升与最小资源预留等机制,避免低优先级作业饥饿。实验表明,CES-Multi在资源利用率、作业等待时间与带宽满足度方面的综合表现优于现有方法,有效增强了CES系统在多作业调度场景中的适用性与综合性能。

\keywordsCN{边缘计算;任务调度;多目标优化;深度强化学习;分布式训练}

\chapter{Abstract}
Collaborative edge computing, as an emerging paradigm, provides technical support for low-latency and high-reliability applications such as autonomous driving and industrial internet by leveraging geographically distributed and heterogeneous edge node resources. However, the heterogeneity of edge environments, the tight coupling between computing and network resources, and the complex network-dependent topologies exhibited by distributed intelligent applications (e.g., distributed training) pose severe challenges to traditional task scheduling mechanisms in terms of resource efficiency and system performance. Current research mainly suffers from two limitations: First, mainstream task dependency models based on directed acyclic graphs (DAGs) struggle to effectively characterize distributed intelligent tasks that may have complex communication patterns such as cycles. Second, existing scheduling schemes generally manage and schedule computing resources and network resources independently, overlooking the inherent trade-off between them, which can easily lead to resource contention and performance bottlenecks. To address the above issues, this paper designs an efficient joint scheduling scheme for computing and network resources in a collaborative edge computing environment. The main contributions are as follows:

This paper designs and implements a collaborative edge scheduling system named CES (Collaborative Edge Scheduler), which, for the first time, jointly manages task-to-node mapping and network bandwidth allocation. Secondly, a joint optimization model for computing and network resources is constructed, which can accurately describe the communication topology and resource requirements of tasks such as distributed training and elastically adapt to dynamic environments. We further propose an intelligent scheduling algorithm based on deep reinforcement learning, which adopts a dual-branch actor-critic architecture and incorporates a heuristic reward function, enabling the system to autonomously learn the globally optimal joint scheduling policy in dynamic environments, thereby optimizing overall performance metrics such as system load balancing and user bandwidth satisfaction. Simulation and system experimental results demonstrate that the proposed algorithm significantly outperforms existing baseline methods in overall performance.

Furthermore, to address scenarios with concurrent multiple distributed learning jobs, this paper proposes the CES-Multi algorithm based on CES. Aiming to improve resource utilization balance, guarantee job bandwidth satisfaction, and reduce average job waiting time, this algorithm integrates a rolling window mechanism, resource-aware greedy sorting, and dynamic starvation prevention strategies under multi-resource constraints of edge nodes. The algorithm dynamically selects candidate jobs through a rolling window, performs comprehensive scoring and sorting based on resource suitability, business priority, and waiting time, and invokes the existing single-job scheduler for resource allocation. Meanwhile, mechanisms such as dynamic priority boosting and minimum resource reservation are employed to avoid starvation of low-priority jobs. Experiments show that CES-Multi outperforms existing methods in terms of resource utilization, job waiting time, and bandwidth satisfaction, effectively enhancing the applicability and comprehensive performance of the CES system in multi-job scheduling scenarios.

\keywordsEN{Edge Computing; Task Scheduling; Multi-objective Optimization; Deep Reinforcement Learning; Distributed Training}