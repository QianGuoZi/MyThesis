\chapter{绪论}
本章为本课题提供了简要介绍。首先概述了研究背景和意义,然后总结了国内外关于协同边缘计算下分布式训练任务调度方法的研究现状,进而详细阐述了本文的主要研究工作及创新点。最后,对本文的组织结构进行了说明。

\section{研究背景和意义}
近年来,随着物联网与人工智能技术的深度融合与规模化应用,计算范式正经历一场深刻的变革,从以云计算为中心的集中式处理模式,逐步向“云-边-端”协同的分布式智能架构演进。在这一过程中,边缘计算作为关键一环,通过将计算、存储与分析能力下沉至网络边缘侧,直接在数据源头或近端进行实时处理,有效缓解了云中心在带宽消耗、服务延迟和隐私安全方面的巨大压力。然而,单一、孤立的边缘节点往往受限于其固有的资源瓶颈(如算力、存储)与物理覆盖范围,难以独立支撑日益复杂、跨地域协作且对实时性有严苛要求的智能应用。在此背景下,协同边缘计算应运而生,它超越了对单一边缘节点的优化,演进为一种旨在实现地理分布广泛、形态异构(包括边缘服务器、网关、车载单元及各类移动设备)的众多边缘节点之间,进行数据、计算、模型与网络资源深度共享与协同调度的新兴范式。该架构通过统一的管控平面,对大规模、异构且地理分散的边缘基础设施进行联合管理与智能编排,从而将离散的边缘节点整合成一张高效的协同网络。这不仅显著提升了系统在服务可靠性、资源利用率和任务完成效率等方面的表现,支持业务的灵活部署与快速扩展,更推动了无处不在的智能服务走向现实。

在上述发展趋势中,分布式训练是实现边缘智能的关键任务,广泛出现在智能驾驶(多车感知协同)、工业互联网(跨厂区质量检测)和智慧城市(分布式视频分析)等前沿领域。此类任务通常将训练数据分散在多个边缘节点上,通过协同执行本地计算并频繁交换模型参数或梯度信息,共同完成全局模型的更新。在协同边缘计算环境下,分布式训练任务的执行涉及多个边缘节点之间的紧密协作,形成一个包含数据并行或模型并行的计算—通信流程。在协同边缘计算环境下,这样一个分布式训练任务的执行,本质上构建了一个逻辑上紧密耦合、物理上分散的计算-通信协同体。其性能表现呈现出鲜明的双重依赖性:既依赖于各参与节点的本地计算能力,更严重依赖于节点间通信链路的带宽、延迟和稳定性。任务的逻辑通信拓扑(如环形、星形等)如何高效、低冲突地映射到底层物理网络拓扑上,直接决定了同步过程的通信开销,进而成为影响整体训练效率(如达到目标精度所需的时间)的决定性瓶颈。

\begin{figure}[htbp]
	\centering
	\includegraphics[scale=0.6]{Fig/task_scene.png}
	\caption{\label{task_scene}协同边缘任务调度场景图}
\end{figure}

这一将分布式训练等复杂任务调度到协同边缘网络中的场景(如图\ref{task_scene}),对资源管理和任务编排提出了一定的挑战。具体而言,该调度场景的核心特征体现在多维资源的耦合性与任务需求的复杂性上。首先,边缘节点并非孤立的计算单元,其计算资源(如CPU、内存)与网络资源(如带宽)紧密耦合,任务对一种资源的消耗往往会影响另一种资源的可用性。其次,任务内部(如分布式训练中的工作节点之间)存在着严格的数据依赖与同步点,形成了复杂的通信拓扑。任务的性能不仅由单个节点的计算能力决定,还会显著受到其内部通信链路所经历的物理网络性能的影响。任务的通信拓扑与底层物理网络拓扑的匹配程度,直接决定了通信开销的大小,从而成为系统整体性能的关键瓶颈。这些因素共同带来了如下核心难点:

(1)分布式训练等任务产生的密集、周期性通信流,在映射到共享的物理网络时,会竞争有限的带宽资源。由于任务逻辑通信拓扑可能非常复杂,不同任务间甚至同一任务内的数据流可能在不经意间共享某条关键物理链路,形成隐蔽的带宽竞争点。这种竞争难以通过局部信息进行准确建模和预测,极易引发局部网络拥塞,造成部分任务同步延迟激增,从而破坏系统整体的负载均衡与性能稳定性。

(2)调度器在决策时面临固有的目标权衡。例如,为了最小化通信开销,理想策略是将频繁通信的任务子单元(如一个分布式训练的Worker)尽可能集中部署在物理距离近、带宽充足的少数节点上。但这往往会导致这些节点形成计算热点,引发排队延迟,并违背了负载均衡的原则。反之,为了最大化计算资源的均衡利用,将任务分散部署到更广泛的节点上,又会显著增加节点间的通信距离与跳数,从而加大通信延迟和带宽压力。这种“计算聚集”与“通信分散”之间的矛盾,使得在协同边缘计算中实现计算与网络资源的联合均衡调度成为一个难以同时优化所有目标的复杂问题。

综上所述,在协同边缘计算环境下,如何设计一种能够同时考虑计算资源与网络资源,并在二者之间实现智能权衡的调度机制,具有广阔的应用前景及重要的研究意义。

\section{研究现状与分析}
\subsection{协同边缘计算任务调度系统}
随着边缘计算的兴起,各大云服务提供商推出了如AWS IoT Greengrass、Azure IoT Edge和KubeEdge等平台,旨在将云计算功能扩展至资源受限的边缘设备,但这些平台仍遵循面向服务的集中式管理架构,对边缘自治和边到边协作支持有限,难以满足自动驾驶、工业物联网等新兴应用对超低时延、高连通性大规模部署和动态可靠服务的需求。在容器编排领域,Kubernetes凭借其领导地位成为调度和管理应用程序的主流工具,但其设计初衷是针对云数据中心环境,假设资源丰富且网络稳定,并未专门考虑边缘原生应用的独特特性,如组件内部依赖性、资源异构性(数据、计算和网络紧密耦合)以及数据局部性,导致在边缘环境中资源利用不足和应用性能不佳。

尽管已有研究通过简化Kubernetes(如MicroK8s、K3s)或扩展其能力到边缘(如KubeEdge、OpenYurt)来适应资源受限环境,但这些方案未改变核心调度逻辑,对应用性能感知不足;而其他工作尝试引入网络感知调度以优化响应时间或任务部署,却忽略了计算资源异构性和数据局部性,且缺乏网络资源(如带宽)的协同编排。此外,在边缘任务调度中,许多研究聚焦于最小化独立任务的平均完成时间或利用启发式、强化学习处理依赖任务,但往往忽视网络流调度,可能引发拥塞,而少数同时考虑任务分配和流程调度的研究也未能优化应用吞吐量或缺乏实际系统实现。

随着边缘计算与分布式训练技术的融合发展,任务调度系统需同时应对计算资源异构性、网络状态动态性及任务依赖复杂性三大挑战,计算与网络资源的联合调度已成为提升分布式训练效率的核心突破口。现有基于Kubernetes的边缘调度方案虽在资源优化方面取得一定进展,但针对分布式训练任务的特性,仍存在显著局限:

FlexiTask调度器构建了多维度资源调度框架,计算资源层面涵盖CPU、内存、磁盘容量及Pod数量,通过短期与中长期资源利用率融合评估节点负载;网络资源层面将带宽纳入调度维度,突破了Kubernetes原生调度的局限。但针对分布式训练任务,网络资源管理中视带宽为静态资源,未考虑参数同步等动态网络依赖的时序性与突发性。
Edge Service框架通过双层架构扩展Kubernetes,计算资源调度可感知节点硬件异构性(CPU/GPU型号、算力差异),网络资源层面通过控制器动态维护节点拓扑与延迟信息。但其计算与网络缺乏协同机制,优化目标仅聚焦节点级负载均衡,未关联算力分配与带宽占用,无法适配分布式训练对全局完成时间、端到端延迟的核心需求。
FAOFE基于Argo工作流引擎,在计算资源调度上,通过预调度阶段的BFS算法生成微服务执行序列,结合边缘节点CPU、内存等计算资源消耗数据优化节点选择;在任务依赖建模上,采用有向无环图(DAG)梳理微服务逻辑关联,解决了Kubernetes调度与工作流任务顺序不一致的问题。但是DAG仅梳理逻辑依赖,未考虑计算节点与参数服务器间的动态网络交互,无法支撑并行训练的资源协同分配。
ENTS作为边缘原生调度系统,实现了计算与网络资源的初步联合编排。ENTS通过Profiler解析任务计算需求,适配异构节点算力,此外,借助Network Controller管理带宽与路由,优化数据传输。但优化目标聚焦吞吐量,未考虑计算资源与网络资源的联动,导致资源配置失衡。

综上,现有调度方案虽在多资源均衡、QoS 感知、任务依赖建模等维度取得突破,但针对分布式训练任务的核心需求(计算-网络协同、动态网络依赖、全局性能优化)仍存在显著不足,缺乏能够同时适配训练任务特性、实现计算与网络资源精细化联合调度的机制。

\subsection{协同边缘计算分布式训练任务调度算法}
当前边缘计算任务调度算法研究普遍采用有向无环图(DAG)对任务进行建模,该模型在处理具有静态依赖关系的任务流时表现良好,例如TF-DDRL框架针对物联网应用的任务依赖建模,以及MARS框架对无人机辅助移动边缘计算系统中计算密集型任务的调度。然而,随着分布式训练等边缘智能应用的深入发展,任务间呈现出多种拓扑结构和动态交互的新特征,使得传统DAG模型在准确刻画此类复杂的数据流向与任务关联时面临挑战。现有研究即便考虑了任务依赖(如TF-DDRL),也往往局限于DAG的静态和无环假设,在适配实时变化的交互关系时效率受限,从而可能影响调度机制在复杂场景下的适应性与资源效率。更关键的是,当前调度研究多集中于独立或简单依赖任务(如BD-TTS中的物联网独立任务、A2C-DRL中的随机到达任务),缺乏对多种网络拓扑下任务协同执行机制的专门设计,这进一步限制了其在分布式协同场景中的适用性。

另一方面,在虚拟网络嵌入(VNE)研究中,尽管考虑到了多种网络拓扑且智能化方法不断涌现,但其资源优化的核心焦点仍存在显著失衡。以CE-VNE、PPO-VNE等为代表的先进算法,虽引入了图卷积网络(GCN)自动提取拓扑特征(CE-VNE, PPO-VNE),并设计了多目标奖励函数以同时优化资源收益与能耗(PPO-VNE),但其优化过程仍侧重于节点侧的计算与内存资源分配。网络带宽、I/O等传输资源在状态设计中常仅作为特征之一(如CE-VNE包含带宽资源,PPO-VNE包含链路可用带宽),而未在奖励机制中被置于与计算资源同等的核心优化地位。RKD-VNE虽创新性地引入了安全性(信任度)作为关键维度,尝试平衡资源利用与安全,体现了多维度优化的趋势,但其决策核心仍围绕节点CPU、带宽等资源的约束满足展开。

这种“重计算、轻网络”的智能优化范式,未能从根本上实现计算、带宽、安全等多维资源的深度协同。例如,其通用的两阶段式“节点-链路”映射动作(CE-VNE、Energy Allocation for V2G等均采用节点选择后接BFS/Dijkstra路径搜索),虽提升了映射效率,但本质上仍将链路带宽优化视为一个被动的、满足约束的后续步骤,而非一个与节点计算资源同步、主动调度的核心决策变量。因此,在设备动态接入、多租户激烈竞争的真实边缘场景中,此类方法仍难以避免由带宽资源碎片化和竞争引发的传输瓶颈,导致整体系统性能受限。

综上所述,当前边缘计算任务调度研究面临双重核心挑战:一是传统DAG模型难以适配具有复杂网络拓扑的分布式训练任务;二是资源优化过程普遍忽视对网络带宽与拓扑的协同调度,导致系统在真实的多租户动态环境中面临严重的性能瓶颈。

\section{本文研究工作及创新点}
针对现有研究在建模分布式训练任务与实现计算-网络协同调度方面的不足,本文以分布式训练等具有复杂通信拓扑的网络依赖型任务为应用背景,旨在设计并开发一种能够联合调度计算与网络资源的智能任务调度系统。
具体而言,本文的核心目标在于:突破传统有向无环图(DAG)模型在建模复杂交互拓扑方面的局限,构建能够准确刻画分布式训练中广泛存在的环状、星型等动态交互拓扑的任务表征与调度框架,从而更真实地反映参数同步、梯度聚合等过程带来的复杂通信行为;同时,将“任务应部署在哪些节点”(计算资源映射)与“应为任务分配多少带宽、选择哪条路径”(网络资源分配)这两个传统上分离的决策过程,深度融合为一个统一的联合优化问题,以系统化地避免因资源分别调度而引发的性能瓶颈。

为实现这一目标,本文构建了能够同时刻画计算约束与网络约束的联合优化模型,该模型不仅考虑了CPU、内存等计算资源的可用性与异构性,还将网络拓扑结构、链路带宽及时延等关键通信因素纳入优化框架。在此基础上,引入深度强化学习方法,使系统能够在动态、异构的边缘环境中,通过与环境的持续交互自主学习全局近似最优的调度策略,实现从感知、决策到执行的闭环优化。最终,我们期望所提出的方案能够为协同边缘计算构建一种资源感知能力强、整体性能优越且具备长期自适应演进能力的一体化任务—资源协同管理机制,从而有效弥补当前研究在应对复杂拓扑结构与多资源协同调度方面的缺陷。此外,为进一步提升系统在实际场景中的实用性,本文还将单作业调度问题系统性地推广至多任务队列调度问题,设计了相应的公平性与效率保障机制,以解决多作业并发执行时的资源竞争与整体调度优化问题。本文的主要贡献如下:

(1)提出了面向协同边缘计算的计算与网络资源联合调度系统CES(Collaborative Edge Scheduler):设计并实现了一个名为CES的协同边缘调度系统。该系统在分布式边缘基础设施中,将计算任务到异构节点的映射与网络带宽的动态分配进行统一编排与闭环管理,实现了对两类紧密耦合资源的协同调度。

(2)构建了联合优化模型并设计了基于深度强化学习的自适应智能调度算法CES-PPO:建立了一个能够同时精准描述计算资源(CPU、内存)需求与网络资源(带宽、拓扑)需求的联合优化模型,并基于此提出了创新的深度强化学习算法。该算法采用双分支演员-评论家网络架构,并在奖励函数中融入启发式知识,使系统能够针对单次分布式训练任务,在动态环境中自主学习全局最优的联合调度策略,从而实现任务映射与带宽分配的协同优化。该模型与算法的有效性在仿真与真实系统实验的多场景测试中得到了验证。

(3)在CES系统基础上,面向多作业并发场景扩展并提出了CES-Multi算法:针对多用户、多分布式学习作业并发的实际场景,在单作业联合调度机制(CES)的基础上,进一步提出了CES-Multi多作业调度算法。该算法以提升资源利用均衡度、保障作业带宽需求满足度及降低作业平均等待时间为目标,融合了滚动窗口、资源感知贪心排序与动态饥饿保护策略,有效解决了多作业间资源竞争的公平性与效率问题,显著增强了CES系统在复杂生产环境中的综合性能与实用价值。
\section{本文组织结构}
本文共分为六个部分,整体组织结构如下:

第一章,绪论。本章首先阐述了协同边缘计算环境下分布式训练任务调度的研究背景与重要意义,分析了当前所面临的核心挑战。随后,从协同边缘计算任务调度系统与协同边缘计算分布式训练任务调度算法两个维度对相关领域的国内外研究现状进行了系统的梳理与深入的评析,指出现有工作的局限性。进而,在此基础上明确了本文的研究目标、研究思路与主要创新点。最后,对全文的组织结构进行了概要说明。

第二章,相关技术基础概述。本章旨在为后续研究提供必要的理论和技术铺垫。首先介绍了协同边缘计算的基本架构与核心特征,然后阐述了分布式训练的任务模型与通信模式,接着分析了任务调度问题的基本框架与关键指标,最后概述了深度强化学习的基本原理及其在资源调度领域的应用范式。这些内容共同构成了理解本文所提方法的基础。

第三章,协同边缘计算任务调度系统CES设计。本章首先明确了面向计算与网络资源联合调度的系统设计目标。随后,详细阐述了所提出的协同边缘调度系统(CES)的整体架构,说明了系统中各核心组件的功能与交互关系。进而,系统性地描述了从任务提交、资源感知、智能决策到最终部署的完整调度流程。本章内容为后续算法的实现与集成提供了系统级的框架支撑。

第四章,基于深度强化学习的联合任务调度算法。首先,对协同边缘环境下单次分布式训练任务的调度问题进行了形式化建模,将其定义为计算与网络资源联合优化的数学问题。随后,详细介绍了基于深度强化学习的智能调度算法设计,包括状态空间、动作空间与奖励函数的设计,以及双分支演员-评论家网络的结构与启发式思想引导训练机制。最后,通过仿真和系统实验与对比分析,验证了该算法在优化系统整体性能方面的有效性与优越性。

第五章,多任务队列调度算法。本章将研究场景从单任务扩展至多作业并发排队调度的更一般情况。首先对多作业调度问题进行了建模,定义了公平性、效率与服务质量等多重优化目标。随后,提出了CES-Multi算法,详细阐述了该算法滚动窗口机制、资源感知的作业排序策略以及动态饥饿保护机制的设计原理与执行流程。最后,通过实验评估了该算法在处理多作业并发时的综合性能。

最后,总结与展望。本章对全文的研究工作与创新成果进行了全面的总结,归纳了所提出的CES系统及其核心算法在解决协同边缘计算资源协同调度问题上的贡献。同时,客观分析了当前研究存在的局限性,并对未来可能的研究方向进行了展望。