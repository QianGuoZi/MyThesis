\chapter{绪论}
本章为本课题提供了简要介绍。首先概述了研究背景和意义,然后总结了国内外关于协同边缘计算下分布式训练任务调度方法的研究现状,进而详细阐述了本文的主要研究工作及创新点。最后,对本文的组织结构进行了说明。
\section{研究背景和意义}
近年来,随着物联网与人工智能技术的深度融合与规模化应用,计算范式正经历一场深刻的变革,从以云计算为中心的集中式处理模式,逐步向“云-边-端”协同的分布式智能架构演进。在这一过程中,边缘计算作为关键一环,通过将计算、存储与分析能力下沉至网络边缘侧,直接在数据源头或近端进行实时处理,有效缓解了云中心在带宽消耗、服务延迟和隐私安全方面的巨大压力。然而,单一、孤立的边缘节点往往受限于其固有的资源瓶颈(如算力、存储)与物理覆盖范围,难以独立支撑日益复杂、跨地域协作且对实时性有严苛要求的智能应用。在此背景下,协同边缘计算应运而生,它超越了对单一边缘节点的优化,演进为一种旨在实现地理分布广泛、形态异构(包括边缘服务器、网关、车载单元及各类移动设备)的众多边缘节点之间,进行数据、计算、模型与网络资源深度共享与协同调度的新兴范式。该架构通过一个统一的管控平面,对大规模、分布广泛且异构的边缘基础设施进行联合管理与智能编排,从而将离散的边缘节点组织成一张高效的边缘协同网络。这不仅能够极大地提升系统在服务可靠性、资源利用率和任务完成效率等方面的表现,支持业务的灵活部署与快速规模化扩展,更推动了无处不在的智能服务真正走向现实。

在上述发展趋势中,分布式训练是实现边缘智能的关键任务,广泛出现在智能驾驶(多车感知协同)、工业互联网(跨厂区质量检测)和智慧城市(分布式视频分析)等前沿领域。此类任务通常将训练数据分散在多个边缘节点上,通过协同执行本地计算并频繁交换模型参数或梯度信息,共同完成全局模型的更新。在协同边缘计算环境下,分布式训练任务的执行涉及多个边缘节点之间的紧密协作,形成一个包含数据并行或模型并行的计算—通信流程。在协同边缘计算环境下,这样一个分布式训练任务的执行,本质上构建了一个逻辑上紧密耦合、物理上分散的计算-通信协同体。其性能表现呈现出鲜明的双重依赖性:既依赖于各参与节点的本地计算能力,更严重依赖于节点间通信链路的带宽、延迟和稳定性。任务的逻辑通信拓扑(如环形、星形等)如何高效、低冲突地映射到底层物理网络拓扑上,直接决定了同步过程的通信开销,进而成为影响整体训练效率(如达到目标精度所需的时间)的决定性瓶颈。


这一将分布式训练等复杂任务调度到协同边缘网络中的场景,对资源管理和任务编排提出了一定的挑战。具体而言,该调度场景的核心特征体现在多维资源的耦合性与任务需求的复杂性上。首先,边缘节点并非孤立的计算单元,其计算资源(如CPU、内存)与网络资源(如带宽)紧密耦合,任务对一种资源的消耗往往会影响另一种资源的可用性。其次,任务内部(如分布式训练中的工作节点之间)存在着严格的数据依赖与同步点,形成了复杂的通信拓扑。任务的性能不仅由单个节点的计算能力决定,还会显著受到其内部通信链路所经历的物理网络性能的影响。任务的通信拓扑与底层物理网络拓扑的匹配程度,直接决定了通信开销的大小,从而成为系统整体性能的关键瓶颈。这些因素共同带来了如下核心难点:

(1)分布式训练等任务产生的密集、周期性通信流,在映射到共享的物理网络时,会竞争有限的带宽资源。由于任务逻辑通信拓扑可能非常复杂,不同任务间甚至同一任务内的数据流可能在不经意间共享某条关键物理链路,形成隐蔽的带宽竞争点。这种竞争难以通过局部信息进行准确建模和预测,极易引发局部网络拥塞,造成部分任务同步延迟激增,从而破坏系统整体的负载均衡与性能稳定性。

(2)调度器在决策时面临固有的目标权衡。例如,为了最小化通信开销,理想策略是将频繁通信的任务子单元(如一个分布式训练的Worker)尽可能集中部署在物理距离近、带宽充足的少数节点上。但这往往会导致这些节点形成计算热点,引发排队延迟,并违背了负载均衡的原则。反之,为了最大化计算资源的均衡利用,将任务分散部署到更广泛的节点上,又会显著增加节点间的通信距离与跳数,从而加大通信延迟和带宽压力。这种“计算聚集”与“通信分散”之间的矛盾,使得在协同边缘计算中实现计算与网络资源的联合均衡调度成为一个难以同时优化所有目标的复杂问题。

综上所述,在协同边缘计算环境下,如何设计一种能够同时考虑计算资源与网络资源,并在二者之间实现智能权衡的调度机制,具有广阔的应用前景及重要的研究意义。

\section{研究现状与分析}
现有的边缘计算任务调度系统仍存在诸多局限:FlexiTask虽然将带宽纳入Kubernetes边缘环境的资源模型,并结合负载预测提升效率,但其视带宽为静态资源,忽略任务间网络依赖,优化目标局限于节点级负载均衡。Edge Service框架支持QoS感知调度,但缺乏网络联合调度与长期全局优化能力,且网络模型过于简化。FAOFE采用有向无环图(DAG)对任务依赖进行建模,但未考虑网络状态,依赖静态规则,难以响应动态变化。ENTS虽深化了网络资源集成并提升吞吐量,但仍未充分实现计算与网络资源的协同优化,且存在目标单一或模型简化问题。

当前边缘计算任务调度算法研究普遍采用有向无环图(DAG)对任务建模,该模型适用于具有任务依赖关系的任务流,并在工作流调度等场景中表现良好。然而,随着边缘智能与分布式协同计算的发展,分布式训练等应用呈现出环状拓扑和动态交互特征,难以用DAG准确描述,导致现有调度机制在适应性和资源效率方面存在明显不足。另一方面,虚拟网络嵌入(VNE)的研究多聚焦于节点计算与内存资源优化,常将网络资源视为次要约束。这种忽略带宽和拓扑调度的做法在多租户、动态变化的边缘环境中易引发资源竞争和传输瓶颈,已成为系统性能的主要短板。
\section{本文研究工作及创新点}
针对上述不足,本文以分布式训练等网络依赖型任务为应用背景,旨在设计并开发一种能够联合调度计算与网络资源的任务调度系统。不再将“任务如何部署”和“网络带宽分多少”当作两个独立的问题,而是作为一个统一的问题来求解。我们构建了同时考虑计算资源和网络资源的联合优化模型,并基于深度强化学习生成全局最优调度策略。该系统致力于为动态、异构的边缘环境提供资源感知能力强、整体性能优越且具备自适应能力的任务—资源协同管理机制,以弥补现有研究在资源协同方面的缺陷。本文的主要贡献如下:

(1)本文首次提出了面向计算资源与网络资源联合调度的 CES 系统:我们设计并实现了一个名为 CES 的系统,能够在分布式边缘基础设施中协同完成计算任务到节点的映射与网络带宽资源的动态分配,实现两类资源的统一管理与调度。

(2)本文构建了计算与网络资源的联合优化模型:建立了同时考虑计算资源(CPU、内存)和网络资源(带宽)的联合优化模型。该模型能够准确描述分布式学习等通信依赖型任务在边缘环境中的通信与计算需求,并弹性适应动态带宽变化,从而实现任务映射策略与带宽分配策略的协同优化。

(3)本文设计了基于强化学习的智能调度机制:采用深度强化学习方法,设计了双分支演员-评论家结构,在奖励函数设计中融入启发式思想,帮助智能体快速学习决策,以实现任务部署与网络资源分配的联合决策。

(4)本文通过仿真实验和系统实验在三种分布式学习任务场景下测试了CES的智能调度机制性能。相较于基准算法,CES能够更好地平衡负载均衡度和带宽满足度,显著提高整体性能。
\section{本文组织结构}


关于\LaTeX{}以及基于\LaTeX{}写作的好处不再赘述。\LaTeX{}的入门资料推荐文献\parencite{_g}以及文献\parencite{_c}。 